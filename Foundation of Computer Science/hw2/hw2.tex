\documentclass{article}
\usepackage[utf8]{inputenc}
\usepackage{amsfonts}
\usepackage{tikz}
\usepackage{amssymb}
\usepackage{amsmath}
\title{FOCS HW2}
\author{Yifan Wang}
\date{September 15th, 2019}

\begin{document}
\maketitle

\newpage
\section*{Problem 3.53}
\subsection*{(a)}
Domain of x should be in
$\mathbb{N}$, $\mathbb{Z}$, $\mathbb{Q}$, $\mathbb{R}$
For example, x=2, $x^2$=4.\\
2 is $\mathbb{N}$, $\mathbb{Z}$, $\mathbb{Q}$, $\mathbb{R}$ at the same time.\\
\subsection*{(b)}
Domain of x should be in $\mathbb{R}$\\
For example, x=$\sqrt{2}$, it's not integer, natural number, or rational.
\subsection*{(c)}
Domain of x and y both should be in
$\mathbb{N}$, $\mathbb{Z}$, $\mathbb{Q}$, $\mathbb{R}$
Because if x in $\mathbb{N}$, $\mathbb{Z}$, $\mathbb{Q}$, $\mathbb{R}$, it will be the same for y.\\
\subsection*{(d)}
Domain of x should be in
$\mathbb{N}$, $\mathbb{Z}$, $\mathbb{Q}$, $\mathbb{R}$
and y should be in $\mathbb{N}$\\
No matter what is x in, y is not negative. So have to be in $\mathbb{N}$.\\





\newpage
\section*{Problem 4.9}
\subsection*{(a)}
\textbf{Direct Proof:}\\ 
$n^3+5 = (n^3+1)+4 = (n+1)(n^2-n+1)+4$.//
Because $n^3+5$ is odd, 4 is even, $(n+1)(n^2-n+1)$ is odd. Only odd multiply by odd is odd. So both $(n+1)$ and $(n^2-n+1)$ are odd. Since n+1 is odd, n is even.\\\\
\textbf{Contraposition Proof:}\\ 
Assume n is odd, which means $n=2\cdot k+1$.\\
$n^3 = (2\cdot k+1)^3 = 8k^3+12k^2+6k+1$\\
So, $n^3+5 = 8k^3+12k^2+6k+6 = 2\cdot (4k^3+6k^2+3k+3)$.\\
As the result, $n^3+5$ is even(not odd) number.\\
 This contraposition is true, so the original statement is true.

\subsection*{(b)}
\textbf{Direct Proof:}\\
If 3 is not dividible n, then n = 3k+1 or 3k+2.\\
$n^2+2 = (3k+1)^2+2$ or $(3k+2)^2+2 = $ $9k^2+6k+3$ or $9k^2+12k+6$\\
These two can be simplify as $3(3k^2+2k+1)$ and $3(3k^2+4k+2)$\\
which are both dividible by 3. So this "if then" relationship is true.\\\\
\textbf{Contraposition Proof:}\\ 
Assume $n^2+2$ is not divisible by 3. $n^2+2 = (n+1)(n-1)+3$ since it is not divisible by 3 and 3 is divisible by 3, (n+1)(n-1) must not be divisible by 3. Neither (n+1) nor (n-1) can be divisible by 3. Because every 3 adjacent numbers should have a number that is divisible by 3 and (n+1)(n-1) are both not. n must be divisible by 3. This contraposition is true so that the original relationship is true.\\

\newpage
\section*{Problem 4.15}
\subsection*{(e)}
(Direct Proof)\\
When n is odd, $n = 2k+1$.\\
$n^2+3n+4 = (2k+1)^2+3(2k+1)+4 = 4k^2+4k+1+6k+3+4$.\\
This can be simplify to $4k^2+10k+8 = 2(2k^2+5k+4)$. Which is even.\\
When n is even, $n = 2k$.\\
$n^2+3n+4 = (2k)^2+3(2k)+4 = 4k^2+6k+4 = 2(2k^2+3k+2)$.\\
Which is also even. Since no matter if n is even or not as long as it is a integer, $n^2+3n+4$ is always even. So, the statement is true. \\

\subsection*{(w)}
(Contraposition Proof)\\
Assume min(a,b) $\geq$ 100. The smallest value of a and b we can get is a = 100, b = 100. In this situation, a $\cdot$ b = 10000 which is not $\leq$ 10000.\\
This contraposition is true. So the original statement is true as well.


\newpage
\section*{Problem 4.26}
\subsection*{(b)}
\textbf{To prove:}\\
Contraposition\\
Prove that $\forall n: $ P(n) is true, Q(n) is false.\\\\
\textbf{To disprove:}\\
Direct prove\\
Using truth table\\
Prove that $\exists n: $ P(n) is false or when P(n) is true, Q(n) is true.\\
\subsection*{(d)}
\textbf{To prove:}\\
Induction\\
Prove that $\forall n: $ P(n) is false or $\forall n: $ P(n) is true and $\forall x: $ Q(x) is true.\\\\
\textbf{To disprove:}\\
Show for counter example\\
Prove that $\exists n: $ P(n) is true and $\exists x: $ Q(x) is false.\\
\subsection*{(f)}
\textbf{To prove:}\\
Show for counter example\\
Prove that $\exists n: $ P(n) is false or $\exists n: $ P(n) is true and $\exists x: $ Q(x) is true.\\\\
\textbf{To disprove:}\\
Show for general object\\
Prove that $\forall n: $ P(n) is true and $\forall x: $ Q(x) is false.

\newpage
\section*{Problem 4.45}
\subsection*{(b)}
\textbf{The solution is (ii).}\\
Proof of (ii):\\
f(n)-1 = (n+3)/(n+1)-1 = ((n+3)-(n+1))/(n+1) = 1/(n+1)\\
When $\varepsilon>$0, $n_\varepsilon$=(2/$\varepsilon$-1)-(2/$\varepsilon$-1)\%1+1\\
2/$\varepsilon$-1 $<$ (2/$\varepsilon$-1)-(2/$\varepsilon$-1)\\
f($n_\varepsilon$)-1 = 2/$n_\varepsilon < $ 2/(2/$\varepsilon$-1+1) = $\varepsilon$\\
f($n_\varepsilon$)-1 $< n_\varepsilon$, which means that the statement is true\\\\
Disproof of (i):\\
f(n) = (n+3)/(n+1) = (n+1+2)/(n+1) = 2/(n+1)+1 $<$ 2/(1+1)+1 = 2\\
when C = 3, $\forall n\in \mathbb{N},$ f(n) $<$ C which is not what (i) states.\\\\
Disproof of (iii):\\
f(n)-2 = (n+3)/(n+1)-2 = ((n+3)-(2n+2))/(n+1) = (-n+1)/(n+1)\\
|f(n)-2| = (n-1)/(n+1)\\
Let's set $\varepsilon$ = 1/3, when n$>$2, (n-1)/(n+1) = 1-2/(n+1) $>$ 1/3\\
This is not what (iii) states.\\\\

\newpage
\section*{Problem 5.7}
\subsection*{(f)}
Let P(n): (1-1/2)(1-1/3)(1-1/4)...(1-1/n)=1/n\\
When n = 2, 1-1/2 = 1/2\\
Let's see if for n $\geq$ 2, P(n) $\rightarrow$ P(n+1)\\
(1-1/2)(1-1/3)(1-1/4)...(1-1/n)(1-1/(n+1))\\
= (1/n)(1-1/(n+1))\\
= (1/n)-(1/(n(n+1)))\\
= ((n+1)/(n(n+1)))-(1/(n(n+1)))\\
= n/(n(n+1))\\
= 1/(n+1)\\
We can see that by induction P(n) $\rightarrow$ P(n+1) is true when n $\geq$ 2\\








\end{document}